\documentclass[a4paper,12pt]{book}
\usepackage[utf8]{inputenc}
\usepackage{graphicx}

\begin{document}
	
Chapter 4
Software Development Methodologies
Programmers subscribe to their chosen methodology
of worship!
Software development methodologies are frameworks used
to plan and control the process of developing software-inten-
sive systems. Over the past several decades, a wide variety of
frameworks have evolved each having specific strengths and
weaknesses. There is no single best system development strat-
egy; good options exist, and no one methodology is good for
all types of projects. Forget about silver bullets and one size
fits all.
As the Software Project Manager, you should have an
in-depth understanding of the alternative methodologies and
their strengths and weaknesses; then you can proceed with
the methodology that best fits your project. Even better, you
can select specific features from the methodologies that are
applicable to your project and create your own hybrid meth-
odology tailored to the needs of your project.
Chapter 4 provides a top-level overview of the most pop-
ular software development approaches, methodologies, mod-
els and standards and is discussed in the following sections:
◾◾ Software Development Process Models: Covers the
most commonly used Software Development Process
Models including a brief description of those models
and their differences (4.1).
◾◾ Software Analysis and Design Models: Covers the
most commonly used Software Analysis and Design
Methodologies including a brief description of each
(4.2).
◾◾ Managing Agile Software Development Projects: Agile
is an umbrella term for multiple project management
methodologies. This section is focused on Agile/Scrum
which is an approach to software development that
focuses on iterative goals set by the Product Owner
through a backlog developed by the Scrum Team,
facilitated by the Scrum Master (4.3).
◾◾ Schedules and Activity Networks: Describes the impor-
tance of a formal program Integrated Master Plan
(IMP) coupled with a related Integrated Master Schedule
(IMS) (4.4).
◾◾ Software Standards: Covers the need for and advan-
tages of software standards to provide consistency dur-
ing system development plus typical software product
and testing levels (4.5).
As the Software Project Manager you must be involved
with the selection of the development methodology because,
once a decision is made, it will have a long-term impact to
your project. The objective of Chapter 4 is to provide an
introduction to that basic understanding providing you with
a foundation for that decision.
4.1 Software Development
Process Models
A Software Development Life Cycle Model should almost
always be used to describe, organize, monitor and control soft-
ware development activities. There is considerable confusion
regarding the differences between some of the development
models. Table 4.1 contains, in alphabetical order, a brief descrip-
tion of the most commonly used Software Development Process
Models. To clarify the intent of each model, they are discussed in
more detail at the indicated locations in the Guidebook.
Managing the Agile methodology is covered in a separate
Section 4.3 since it is important enough to warrant its own
section (actually, it is worth more than a section, but this
Guidebook is intended to be a compendium and not a huge
tome). Your project must select the strategy (or strategies)
appropriate to the system you are developing with empha-
sis on the availability of requirements. The process selected
should be defined in your Software Development Plan (SDP)
as described in Subsections 1.5.1 and 5.1.1. More than one
49
50 ◾
Project Management of Large Software-Intensive Systems
Table 4.1
SoftwareProcess Model
Agile Model
Evolutionary
Model
Incremental
Model
Iterative Model
Prototyping
Spiral Model
Unified Model
Waterfall
Model
Development Process Models
Brief Description
The term Agile covers a number of methods and approaches for software development. It is focused
on frequent delivery and testing of small portions of the full system. Agile methods are based on the
principles of human interaction, where solutions evolve through collaboration of small self-
organizing, cross-functional teams. Agile is conceived to be flexible and more quickly able to
respond to changes than traditional models (see 4.3).
With this model, the software product is developed in a series of builds (blocks or stages) with
increasing functionality. The requirements are defined for each evolutionary build as that build is
developed. This is a “build-a-little, test-a-little” development process model that can provide an early
operational capability for a portion of the entire system, and it is highly amenable to systems with
evolving requirements (see 4.1.1).
This model requires that all of the requirements are defined up front; the software product is then
developed in a series of builds with cumulative increasing functionality. A portion of the software
product is built and tested—one small increment at a time. This is also a “build-a-little, test-a-little”
approach that can provide an early operational capability for a portion of the entire system (see
4.1.2).
Not really a software development model but more of a quality improvement approach where a fully
developed and delivered systems is periodically updated and improved with each new release of the
product. After a system is developed using one of the other development models, the iterative
approach is used to improve its quality (see 4.1.3).
This development approach involves building an early experimental portion of a system to better
understand the requirements and interfaces, to test throughput speeds, develop environment
testing, etc. Normally the product produced is built fast, without sufficient documentation, and not
designed to be maintainable, so it normally cannot be used as the final product, but there are
exceptions (see 4.1.4).
The Spiral Model is a risk-driven software development process that has two main features: (1) A
cyclic approach that grows a system’s functionality and implementation incrementally while focusing
on decreasing its degree of risk; and (2) A set of anchor point milestones for insuring stakeholder
commitment to acceptable system solutions. Implementations using this model are often done in
conjunction with the Evolutionary Model (see 4.1.5).
A variation of the Spiral Model is the Unified Process exemplified by the IBM Rational Unified
Process® (RUP®). RUP is an iterative software development framework. However, it is not a single
prescriptive process but an adaptable process framework intended to be tailored by selecting
elements of the process applicable to each user. It involves an underlying Object-Oriented Model
using the Unified Modeling Language (see 4.1.6).
A linear sequential software development model that requires all functionality and design
requirements to be defined up front and each development activity to be completed before the next
activity begins, although some overlap is allowed. The entire software product is not available until
the last testing activities are completed (see 4.1.7).
Software Development Life Cycle Model may be needed for
different types and applications of software used.
4.1.1 Evolutionary Software
Acquisition Strategy
As defined in the dictionary, the word “evolution” refers to
a “process of continuous change from a lower to a higher,
more complex, or better state.” In that context, evolutionary
software development could be considered a generic strategy,
applicable to other development methodologies where the
functionality of the software product “evolves” or is refined
during the course of each increment or iteration. Key to the
success of evolutionary acquisition is continuous customer
involvement in the articulation, validation and prioritiza-
tion of system requirements over the life of the development
process. An example development scenario following the
Evolutionary Model is shown in Figure 4.1.
Software Development Methodologies ◾
Evolutionary Software Development Model—example.
The figure shows three builds (B1-B2-B3) where each
build adds more functionality. The requirements are defined
or updated for each build. In this example, the requirements
analysis for Build-2 starts during the Build-1 design, and
the Build-3 requirements analysis starts during the Build-2
design. Software Item Qualification Testing (SIQT) takes
place at the end of Build-3 followed by progressive integra-
tions with Hardware Items (HI), other Software Items (SI)
and then with the other subsystems. The Evolutionary Model
is time-tested; consider it seriously for medium to large soft-
ware-intensive system developments.
NOTE: Increment versus Iterate. Before discussing the
Incremental Model and the Iterative Model, their differences
should be noted because they are easily confused, and some-
times the two terms are used interchangeably. According to
(Cockburn, 2008), increment fundamentally means add onto
whereas iterate generally means re-do or rework. They are fun-
damentally different models, each serves a different purpose,
and they need to be managed differently. Cockburn states:
“The development process, feature set, and product quality
all need constant improvement. Use an incremental strategy,
with reflection, to improve the first two. Use an iterative strat-
egy, with reflection, to improve the third.” Please note that
the word “iterate,” “iterative” or “iteration” with a small case
“i” refers to the common meaning of the word: “repetition.” A
capital “I” is used when referencing the model name.
4.1.2 The Incremental Model
The Incremental Model is a method of developing software
where the product is developed in a series of builds with
increasing functionality. A portion of the software product is
built, tested and integrated—one small increment at a time.
This is a build-a-little, test-a-little approach that can provide
an early operational capability for a portion of the entire
system. The product is defined as finished when it satisfies all
of the requirements that were defined up front.
The Incremental Model combines the elements of the
Sequential Model (Waterfall; see 4.1.7) with the iterative phi-
losophy of prototyping (see 4.1.4). The biggest problem with
this approach is that it presumes you can define all of the
software requirements up front. Assuming there will be no
changes during development is unrealistic unless there is an
edict to freeze the requirements for the current version. A vari-
ation of this model could include the provision for some feed-
back and changes during the build process. Regression Testing
is an important element of the incremental build model.
A big advantage of the Incremental Model is that there
is a working model of a portion of the system at an early
stage of development making it easier to find functional or
design flaws. Finding issues at an early stage of development
is extremely cost-effective (see Subsection 6.5.1). Aside from
needing requirements to be defined up front, a disadvantage
of the Incremental Model is that it is typically applicable
only to large software development projects because it may
be hard to break a small software system into even smaller
serviceable increments.
4.1.3 The Iterative Model
The Iterative Model can be defined as delivering an entire
system and then changing the functionality of the system
as needed during each subsequent release of the product.
This model involves a cyclical process. A cornerstone of the
Iterative Model is the fact that it is often very difficult to
know upfront exactly what the customer really wants, so the
process builds in as many learning opportunities as possible.
The learning you gain from the iterations provides construc-
tive information to the next iteration. This learning can come
from end-users, Testers or the Developers themselves.
52
◾
Project Management of Large Software-Intensive Systems
When the Iterative Model is used, the first delivery is
likely to be a “core product.” The subsequent iterations are
the supporting functionalities or the add-on features that the
customers want. The product is designed, implemented and
tested as a series of incremental deliveries until it has all the
functionality required by the users.
For example, a shrink wrap company may want to get a
product to market within a prescribed time frame, regardless
of its inferior quality, and then rely on future releases to fix
the bugs and improve its quality. They may call this approach
a “development model” because that is the iterative approach
they follow. It can be argued that this is not a development
model because the subsequent upgraded improvements could
be considered the maintenance phase.
Another perspective of the Iterative Model can be
explained with the analogy of a portrait painter who provides
the customer initially with a very simple sketch; the customer
decides the head is facing in the wrong direction so at this
point it is easy for the painter to make that adjustment. The
painter then adds a little more detail and presents it again to
the customer for review. The customer makes recommended
changes and the “iteration” starts over again and continues
until the customer approves what will be the final version.
For a software project where requirements are unknown, this
approach may work, however, for a large or mega system this
analogy does not generally apply.
The key to successful use of an iterative Software
Development Life Cycle is rigorous validation of the new
requirements, and verification and testing of each version of
the software product against those requirements. As the soft-
ware evolves through successive release cycles, tests have to
be repeated and extended to verify each version of the soft-
ware. Software project managers often consider each itera-
tion a separate project.
4.1.4 Prototyping
Prototyping is the process of building an early experimental
model of a defined system that provides partial function-
ality of the final product. Prototyping allows customers to
evaluate development approaches and “try them out” before
implementation and to better understand the requirements.
Developing prototypes is an extra cost element but provides
useful benefits especially for systems having a high level of
user interactions such as online systems where users need to
fill out forms or go through various screens before data is
processed. A prototype can be used very effectively to give
the exact look and feel before the actual software is devel-
oped. The major software prototyping types used widely are:
analysis to build a prototype. Once the actual require-
ments are understood, the prototype is discarded, and
the actual system is developed with a clearer under-
standing of user requirements.
◾◾ Evolutionary Prototyping: Evolutionary prototyping,
also called breadboard prototyping, is based on building
actual functional prototypes with minimal functional-
ity in the beginning but built up incrementally to even-
tually deliver the entire system with full functionality.
◾◾ Integration Prototyping: Refers to building independent
multiple functional prototypes of the various subsys-
tems and then integrating all the available prototype
subsystems to form the complete system.
The trap with the last two types is the tendency to delay
or avoid developing the documentation needed by the final
product and the difficulty, and cost, of recreating the docu-
mentation later.
Prototyping addresses the inability of many customers
to specify their exact information needs and the difficulty
of system analysts to understand the user’s environment, by
providing the customer with a tentative system for experi-
mental purposes at the earliest possible time. The major prob-
lem with prototyping is that too often a prototype becomes a
quick but inefficient approach to meeting customer require-
ments. The code will most often need major reconstruction,
and most Programmers do not want to throw away their
code so you may end up with patchwork code that is difficult
to maintain.
Another problem with prototyping is the frequent inter-
active interface with the customer during prototyping may
lead to requirements scope creep. Every time you believe you
are finished there can be new improvements or new func-
tionality proposed to make it better (see Gold Plating 5.1.5).
You can avoid this if you plan a specific number of iterations
and issue a cut-off date for adding new functionality to the
system.
4.1.5 The Spiral Model
The Spiral Model is a risk-driven Evolutionary Software
Development Model that couples the iterative nature of pro-
totyping with the controlled and systematic aspects of the
Linear Sequential Model where the software is developed
in a series of incremental releases. Early releases typically
produce a simple prototype, however, during later iterations
more complex functionalities are added. The Spiral Model
is a risk-driven Software Development Process Model that has
two main features:
◾◾ Throwaway/Rapid Prototyping: Throwaway prototyp-
ing is also called rapid or close-ended prototyping.
This type of prototyping needs minimal requirement
◾◾ A cyclic approach that grows a system’s functionality
and implementation incrementally while focusing on
decreasing its degree of risk

\end{document}